\documentclass[../ComplexAnalysis_Notes.tex]{subfiles}

\begin{document}
\chapter*{Lecture 1} %Set chapter name
\addcontentsline{toc}{chapter}{Lecture 1} %Set chapter title
\setcounter{chapter}{1} %Set chapter counter
\setcounter{section}{0}
\setcounter{equation}{0}
\setcounter{figure}{0}


\section{Introduction}

Our objective is to study functions $f : \C \to \C$. We know that as metric spaces $\C$ and $\R^2$ are isometric, with the natural map $(x,y) \mapsto x + i y$ being an isometry, but then what is the difference between analysis in $\R^2$ and analysis in $\C$? The difference arises because $\C$ is a field while $\R^2$ is not a field, thus we have a notion of multiplication and division in the complex plane.

Before going into further details we recall some of the obvious observations that one can make,
\begin{enumerate}
  \item (Triangle inequality). \index{Triangle inequality} $||z_1| - |z_2|| \leq |z_1 - z_2|$.
  \item $|z| \geq \max \{ |x|, |y| \}$ where $z = x + iy$.
  \item $\{z_n\}_{n \in \N} \subseteq \C$ is a Cauchy sequence\index{Cauchy sequence} if and only if $\{(x_n,y_n)\}_{n \in \N} \subseteq \R^2$ is a Cauchy sequence, which is equivalent to $\{x_n\}_{n\in\N}$ and $\{y_n\}_{n\in\N}$ are Cauchy sequences.
  \item Let $f : \C \to \C$ be a function. Then $f$ is continuous (or limit exists) at a point $z_0=x_0 + i y_0$ if and only if $f : \R^2 \to \R^2$ viewed as a function from the real plane to the real plane is continuous (or limit exists) at $(x_0,y_0)$.
\end{enumerate}
We haven't yet clarified how the analysis of $\C$ differs from the analysis of $\R^2$, the fact that $\C$ is a field gives us that $\frac{f(z) - f(z_0)}{z - z_0} \in \C$ for all $z \neq z_0$. Thus we can define the derivative of $f : \C \to C$ at $z_0$ as the complex number obtained by taking the limit $\lim_{z \to z_0} \frac{f(z) - f(z_0)}{z - z_0}$ (provided the limit exists). Thus the derivative of $f : \C \to C$ at $z_0 \in \C$ is a complex number, while the derivative of $f : \R^2 \to \R^2$ (that is, the total derivative) is a $2 \times 2$ matrix.

This raises the following question let $f = u + i v$, then if we view $f = (u,v) : \R^2 \to \R^2$ we know
\[
  J_f(x_0,y_0) = Df(x_0,y_0) = \begin{bmatrix}
    u_x(x_0,y_0) & u_y(x_0,y_0) \\ v_x(x_0,y_0) & v_y(x_0,y_0)
  \end{bmatrix},
\]
is there any relation between $f'(z_0)$ (if it exists) and $J_f(x_0,y_0)$?

\smallskip

\textbf{Homewok.} (another representation of $\C$). Let
\begin{equation}\label{eq1:lec1}
  M = \left\{ \begin{pmatrix} x & -y \\ y & x \end{pmatrix} \mid x,y \in \R  \right\} \subseteq M_2(\R)
\end{equation}
Show that $M$ is a field under matrix multiplication and is in fact isomorphic to $\C$.

\smallskip

The above assignment suggests there must be some representation of $f'(z_0)$ in terms of the Jacobian matrix $J_f(x_0,y_0)$, indeed there is some relation which we will discuss in a while.

\textbf{Notation.} $B_r(z_0) = \{ z \in \C \mid |z - z_0| < r \}$.

\begin{Def}{Holomorphic Functions.}{}\index{Holomorphic Function}
  Let $\mO$ be an open subset of $\C$, and let $f : \mO \to \C$ be a function and $z_0 \in \mO$. We say that $f$ is $\C$-\textbf{differentiable at} $z_0$ or \textbf{holomorphic at} $z_0$ if
  \[
    \lim_{z \to z_0} \frac{f(z) - f(z_0)}{z - z_0} =: f'(z_0) \mbox{ exists.}
  \]
  And we will say $f$ is \textbf{holomorphic on} $\mO$ if $f$ is holomorphic at every point $z \in \mO$. We will denote by
  \[
    \Hol(\mO) = \{ f : \mO \to \C \mid f \mbox{ is holomorphic}\}.
  \]
  Then $\Hol(\mO)$ forms an algebra over $\C$.
\end{Def}

\begin{Lem}{Some Immediate Observations.}{}
  Let $f,g : \mO \to \C$ be holomorphic at $z_0$, then
  \begin{enumerate}
    \item $f$ is continuous at $z_0$.
    \item $(\alpha f + g)'(z_0) = \alpha f'(z_0) + g'(z_0)$ for all $\alpha \in \C$.
  \end{enumerate}
\end{Lem}

\begin{Eg}{}{}
  Some examples of holomorphic functions are $f(z) = z$, $f(z) = $ constant and $f(z) = z^2$, while $f(z) = \bar{z}$ is not a holomorphic function. Note that in $\R^2$ the function $f(z) = \bar{z}$ corresponds to the function $f(u,v) = (u,-v)$. But then we get that $Df(u,v) = \begin{bmatrix}
      1 & 0 \\ 0 & -1
    \end{bmatrix} \notin M$ (where $M$ is defined in equation \ref{eq1:lec1}). If we now consider the function $f(z) = z^2$, then in $\R^2$ it corresponds to the function $f(x,y) = (x^2 - y^2, 2xy)$ then $J_f(x,y) = \begin{bmatrix}
      2x & -2y \\ 2y & 2x
    \end{bmatrix} \in M$.
\end{Eg}

The above example gives us the motivation to answer the problem we had raised earlier: \textit{how are the complex derivative and the Jacobian\index{Jacobian} matrix related?}

\section{Holomorphic versus Differentiable Functions.}\index{Differentiable Function}

For this discussion we will let $f = u + iv : \mO \to \C$, let $z_0 = x_0 + i y_0$. Suppose $f$ is holomorphic at $z_0$, and let $\alpha = a + i b = f'(z_0)$. We then define the function for all $z \in B_r(z_0)$
\begin{align*}
  R(z) & = f(z) - f(z_0) - \alpha (z-z_0)                                                                                                 \\
       & = \underbrace{[u(z) - u(z_0) - a(x-x_0) + b(y-y_0)]}_{R_1(z)} + i \underbrace{[v(z) - v(z_0) - b(x-x_0) - a(y - y_0)]}_{R_2(z)}.
\end{align*}
Now recall that $f : \R^2 \to \R^2$ is differentiable at $(x_0,y_0)$ if and only if
\[
  \frac{R(x,y)}{\| (x,y) - (x_0,y_0) \|} \to 0 \mbox{ as } (x,y) \to (x_0,y_0).
\]
But we have
\[
  \frac{R(z)}{|z-z_0|} = \frac{R_1(z)}{|z-z_0|} + i \frac{R(z)}{|z-z_0|},
\]
and we also know that $f$ is holomorphic at $z_0$ hence we get that
\[
  \lim_{z \to z_0} \frac{R(z)}{|z-z_0|} = 0 \Longleftrightarrow \lim_{z \to z_0} \frac{R_1(z)}{|z-z_0|} = \lim_{z \to z_0} \frac{R_2(z)}{|z-z_0|} = 0.
\]
Thus it is equivalent to saying that $u,v : \mO \to \R$ are differentiable at $(x_0,y_0)$ and we further have
\begin{align*}
   & a = u_x = v_y    \\
   & b = v_x = - u_y.
\end{align*}

\begin{Thm}{Cauchy Riemann Equations}{cre}\index{Cauchy Riemann Equations}
  Let $f := u + iv : \mO \to \C$ be a function and $z_0 \in \mO$. Then $f$ is holomorphic at $z_0$ if and only if $u,v : \mO \to \R$ is differentiable at $z_0$ and $u_x = v_y$ and $u_y = - v_x$. These are called the Cauchy Riemann Equations, thus we have
  \begin{align*}
    u_x = v_y, \ u_y = -v_x \quad \mbox{ and } \quad f'(z_0) = u_x(z_0) + i v_x(z_0).
  \end{align*}
\end{Thm}

\end{document}