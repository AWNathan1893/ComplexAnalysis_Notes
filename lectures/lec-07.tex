\documentclass[../ComplexAnalysis_Notes.tex]{subfiles}

\begin{document}
\chapter*{Lecture 7} %Set chapter name
\addcontentsline{toc}{chapter}{Lecture 7} %Set chapter title

\setcounter{chapter}{7} %Set chapter counter
\setcounter{section}{0}
\setcounter{equation}{0}
\setcounter{figure}{0}

In this lecture, we will complete our discussion of power series and consider analytic functions over the complex plane. We will then prove the equivalence of analytic and holomorphic functions - deriving the Cauchy integral formulae in the process. This is truly marvelous result with no immediate analogue in real analysis, and will be extremely useful in the discussions to follows.

\section{Power Series continued}
We started our discussion of power series over $\C$ in the previous lecture. We give some simple illustartions before moving on.

\begin{Eg}{}{}
Consider the power series given by $\sum_{n=1}^\infty (1+(-1)^n)z^n$. Then the radius of convergence for the same is given by
\[\frac{1}{R} = \limsup_{n \to \infty} \sqrt[n]{(1+(-1)^n)} = 1\]
Note that $\lim{\mid \frac{a_{n+1}}{a_n} \mid}$ does not exist here, so one cannot use the ratio test. 
\end{Eg}

\begin{Eg}{Exponential function}{}

Consider the power series $\sum_{n=1}^\infty \frac{z^n}{n!}$. Evidently, $\lim {\mid \frac{a_{n+1}}{a_n} \mid} = 0$, and thus the power series defines a function over $\C$. We call this the exponential function and denote it by $\exp{z}$. Also for $z_1, z_2 \in \C$, we have

\begin{align*}
  \exp\{z_1\}\exp\{z_2\} &= \left(\sum_{k=0}^{\infty} \frac{z_1^k}{k!}\right)\left(\sum_{l=0}^{\infty} \frac{z_2^l}{l!}\right)\\
  &= \sum_{n=0}^{\infty} \sum_{r=0}^n \frac{z_1^r z_2^{n-r}}{r!(n-r)!}\\
  &= \sum_{n=0}^{\infty} \frac{(z_1 + z_2)^n}{n!} = \exp\{z_1 + z_2\}\\
\end{align*}

Note that the reordering of sums in justfied as the series is absolutely convergent. Using this multiplicative property, along with continuity of power series, one can show that
\[\exp{z} = e^z = \sum_{n=0}^{\infty} \frac{z^n}{n!},\]
where $e := \exp\{1\} = \sum_{n=0}^{\infty} \frac{1}{n!}$. We will talk about this in more detail in the upcoming lectures.
\end{Eg}

\begin{Eg}{Sine and Cosine functions}{}
We define the sine and cosine functions over $\C$ using the complex power series analogous to the Taylor series expansion of their real counterpart. In particular,

\begin{align*}
\sin(z) &= \sum_{n=0}^\infty \frac{(-1)^n}{(2n+1)!}z^{2n+1}, \text{ and}\\
\cos(z) &= \sum_{n=0}^\infty \frac{(-1)^n}{(2n)!}z^{2n}
\end{align*}
Note that these power series absolutely converge over $\C$, and thus define entire functions. We also have the Euler's identity

\begin{align*}
e^z = \sum_{n=0}^{\infty} \frac{z_1^n}{n!} &= \sum_{n=0}^\infty \frac{(-1)^n}{(2n)!}z^{2n} + i\sum_{n=0}^\infty \frac{(-1)^n}{(2n+1)!}z^{2n+1}\\
&= \cos(z) + i\sin(z)
\end{align*}

\end{Eg}

We now comment on the differentiability of functions defined by power series. The exactly analogous result holds for real power series as well, and the proof is almost identical.

\begin{Thm}{}{}

Let $R$ be the radius of convergence of the power series $f(z) := \sum a_n z^n$, where $z \in B_R(0)$. Then $f \in \Hol(\mO)$ and the derivative $f^\prime(z) = \sum na_n z^{n-1}$ is given by the correponding term-by-term differentiation, for $z \in B_R(0)$.

\end{Thm}

\begin{proof}
First we compute the radius of convergence $R^\prime$ of the derived power series \(\sum na_n z^{n-1}\). Evidently,
\[\frac{1}{R^\prime} = \limsup_{n \to \infty} \sqrt[n]{na_n} = \limsup_{n \to \infty} \sqrt[n]{a_n} = \frac{1}{R}\]
Thus the derived series defines a function $g: B_R(0) \to \C$, such that
\[g(z) = \sum_{n=0}^\infty na_n z^{n-1}\]
Fix $z \in B_R(0)$ and correspondingly choose$\delta > 0$ such that $|z|+\delta < R$. Then for $|h| < \delta$,
\[\frac{f(z+h) - f(z)}{h} - g(z) = h \sum_{n=2}^\infty a_n p_n(z,h),\]
where $p_n(z,h) = \sum_{k=2}^n \binom{n}{k}h^{k-2}z^{n-k}$. The proof now follows as
\begin{align*}
\Bigg\rvert\frac{f(z+h) - f(z)}{h} - g(z)\Bigg\rvert &\leq |h| \sum_{n=2}^\infty |a_n| |p_n(z,h)|
\leq |h| \sum_{n=2}^\infty |a_n| p_n(|z|,\delta)\\
&\leq \frac{|h|}{\delta^2} \sum_{n=2}^\infty |a_n|(|z|+\delta)^n {\longrightarrow}\,0 \text{ as } h \to 0\\
\end{align*}
\end{proof}

Thus, if $R$ be the radius of convergence of a power series $f(z) := \sum a_n z^n$, where $z \in B_R(0)$, then $f \in C^\infty(B_R(0))$, where the power series coefficients are given by the corresponding term-by-term differentiation. Also, the coefficients $\{a_k\}$ can be computed from the derivatives of the function $f$ at $z=0$ as
\[f^{(k)}(z) = \sum_{n=k}^\infty n(n-1)\dots(n-k+1)a_{n} z^{n-k}\]
\[\therefore  a_k  = \frac{f^{(k)}(0)}{k!}\]

\section{Analytic Functions}

Polynomials constitute the simplest examples of functions holomorphic on a given domain. The next, most prototypical examples are those which are locally generated by power series.

\begin{Def}{Analytic function}{}
Let $f:\mO \to \C$ be a continuous function, where $\mO$ is open in $\C$. We say that $f$ is analytic at $z_0 \in \mO$ if there exists $r>0$ such that $f$ can be expressed as a convergent power series on $B_r(z_0)$. If this holds for all points $z_0 \in \mO$, then we say $f$ is analytic on $\mO$.
\end{Def}

From the results of the previous section, it is clear that analytic functions are holomorphic, and in fact, infinitely differentiable. For $f:\mO \to \C$ analytic at $z_0$, the local power series representation is

\[f(z) = \sum_{k=0}^\infty \frac{f^{(k)}(z_0)}{k!}(z - z_0)^k\]

The converse is decidedly false in the real case. Consider for instance, the function $f \in C^\infty(\R)$, defined by
\[
f(x) = 
\begin{cases}
e^{-\frac{1}{x^2}} & \text{ if } x>0\\
0 & \text{ if } x \leq 0\\
\end{cases}
\]

It is easy to check that all derivatives of $f$ vanish at $x=0$, but $f$ is not identically zero in any neighbourhood of $x=0$. Thus, $f$ is not analytic at $x=0$. However this result has a positive answer for holomorphic functions; it truly is one of the most remarkable results of complex analysis.

\begin{Thm}{}{}
Let $\mO$ be an open subset of $\C$, and $f \in \Hol{\mO}$. Consider $z_0 \in \mO$ and $\delta>0$ such that $\overline{B_{\delta}(z_0)} \in \mO$. Then, for $C = C_r(z_0)$, we have $f(z) = \sum a_n (z - z_0)^n$, where
\[a_n = \frac{f^{(n)}(z_0)}{n!} = \frac{1}{2\pi i}\oint_{C} \frac{f(\zeta)}{(\zeta - z_0)^{n+1}}d\zeta\]
\end{Thm}

\begin{proof}
By Cauchy integral formula for a circular contour, $\forall\;z\in B_r(z_0)$,
\[f(z) = \frac{1}{2\pi i}\oint_{C} \frac{f(\zeta)}{\zeta - z}d\zeta\]
Now, the following series converges uniformly for $\zeta \in C\setminus \{z_0\}$.
\begin{align*}
\frac{1}{\zeta-z} &= \frac{1}{\zeta - z_0}\frac{1}{1- \left(\frac{z-z_0}{\zeta - z_0}\right)}\\
&= \frac{1}{\zeta - z_0} \sum_{n=0}^\infty \left(\frac{z-z_0}{\zeta - z_0}\right)^n
\end{align*}
Then, as Reimann integration behaves well under uniform convergence over compact domains, we have
\begin{align*}
f(z) &= \frac{1}{2\pi i} \oint_{C} \frac{f(\zeta)}{\zeta - z_0}\sum_{n=0}^\infty \left(\frac{z-z_0}{\zeta - z_0}\right)d\zeta\\
&= \frac{1}{2\pi i}\sum_{n=0}^\infty (z-z_0)^n \oint_{C} \frac{f(\zeta)}{(\zeta - z_0^{n+1})}d\zeta\\
&= \sum_{n=0}^\infty a_n (z-z_0)^n\\
\end{align*}
Thus $f$ is analytic over $\mO$, with
\begin{equation}\label{cauchyintfor}
f^{(n)}(z_0) = \frac{n!}{2\pi i}\oint_{C} \frac{f(\zeta)}{(\zeta - z_0)^{n+1}}d\zeta
\end{equation}
\end{proof}
The equations \eqref{cauchyintfor} are known as Cauchy integral formulae. This theorem sets apart complex analysis from our familiar terrain of real analysis - and has many striking consequences. We will see many applications of this wonderful theorem in the next lecture. For now, we have the simple corollary:
\begin{Cor}{}{}
For $\mO$ an open subset of $\C$, $f \in \Hol(\mO)$ if and only if $f$ is analytic on $\mO$.
\end{Cor}
\end{document}