\documentclass[../ComplexAnalysis_Notes.tex]{subfiles}

\begin{document}
\chapter*{Lecture 11} %Set chapter name
\addcontentsline{toc}{chapter}{Lecture 11} %Set chapter title
\setcounter{chapter}{11} %Set chapter counter
\setcounter{section}{0}
\setcounter{equation}{0}
\setcounter{figure}{0}

Recall in the last lecture we discussed about Laurent series
\[
  \sum_{n \in \Z} a_n (z - z_0)^n,
\]
and which converges uniformly on compact subsets of the annulus
\[
  A_{R_1, R_2}(z_0) = \qty{ z \in \C \mid R_1 < \abs{z - z_0} < R_2 }.
\]
where \(R_1\) and \(R_2\) are given by,
\begin{equation} \label{eq:lsorc}
  R_1 = \limsup_{n \to \infty} \sqrt[n]{\abs{a_{-n}}}, \text{ and } R_2 = \frac{1}{\limsup_{n \to \infty} \sqrt[n]{\abs{a_n}}}.
\end{equation}

Natural question arises, if a function \(f \in \Hol(A_{R_1, R_2}(z_0))\), then does it have a Laurent series expansion? The answer is yes, and we will prove this in this lecture.

\

\textbf{Remark.} Consider the Laurent series \(f(z) = \sum_{n \in \Z} a_n (z - z_0)^n\) on \(A_{R_1, R_2}(z_0)\). Then, we fix \(r \in (R_1, R_2)\) and consider,
\begin{align*}
  \int_{C_r(z_0)} f(z) \dd{z}
   & = \sum_{n = -\infty}^{\infty} a_n \int_{C_r(z_0)} (z - z_0)^n \dd{z}                                                               \\
   & = \sum_{n = 0}^{\infty} a_n \underbrace{\int_{C_r(z_0)} (z - z_0)^n \dd{z}}_{0,\, \text{by Cauchy's Theorem}}                      \\
   & \quad + a_{-1} \int_{C_r(z_0)} \frac{1}{z - z_0} \dd{z}                                                                            \\
   & \quad + \sum_{n = 2}^{\infty} a_{-n} \underbrace{\int_{C_r(z_0)} (z - z_0)^{-n} \dd{z}}_{0,\, \text{by Cauchy's Integral Formula}} \\
   & = 2\pi i\, a_{-1}.
\end{align*}

In other words, we have,
\[
  a_{-1} = \frac{1}{2 \pi i} \int_{C_r(z_0)} f(z) \dd{z}.
\]
This coefficient is very important, and carries some meaning. It is called the \textit{residue} of \(f\) at \(z_0\). We will see this for general \(\C \to \C\) functions in later lectures.

Following the above computation, we can also compute the other coefficients, i.e., for all \(n \in \N\),
\[
  a_n = \frac{1}{2 \pi i} \int_{C_r(z_0)} \frac{f(z)}{(z - z_0)^{n + 1}} \dd{z}.
\]
for every \(r \in (R_1, R_2)\). In particular, if \(f\) is a Laurent series, then it is holomorphic on \(A_{R_1, R_2}(z_0)\), and the coefficients are given by \eqref{eq:lsorc}.

\begin{Def}{Multiply Connected}{mulconn}
  A domain in \(\C\) is said to be multiply connected if it is not simply connected.
\end{Def}

We will be interested in \(n\)-holed domains to compute integrals.

\begin{Thm}{}{nholedsum}
  Let \(\Omega\) be a \(n\)-connected domain, and \(\gamma\) be a simple closed curve enclosing all the \(n\) holes. Let \(\gamma_j\) be a simple closed curve enclosing the \(j^{\text{th}}\) hole, for \(j = 1, \ldots, n\). Then, for any \(f \in \Hol(\Omega)\), we have
  \[
    \int_{\gamma} f(z) \dd{z} = \sum_{j = 1}^{n} \int_{\gamma_j} f(z) \dd{z}.
  \]
\end{Thm}
\begin{proof}
  Proving for \(n = 2\) suffices as the general case follows by induction. We divide \(\Omega\) as in the following figure.
  
  \begin{figure}[H]
    \centering
    \includegraphics[scale=1]{../figures/two_circle}
    \caption{Two circles \(\gamma_1\), \(\gamma_2\) enclosing the holes.}
    \label{fig:two_circle}
  \end{figure}
  
  Then by this construction, we have
  \begin{align*}
    \int_{\gamma} f(z) \dd{z}
     & = \int_{\gamma_1} f(z) \dd{z} + \int_{\gamma_2} f(z) \dd{z} + \underbrace{\int_{\gamma_3} f(z) \dd{z}}_{0, \text{ by Cauchy's Theorem}} + \underbrace{\int_{\gamma_4} f(z) \dd{z}}_{0, \text{ by Cauchy's Theorem}} \\
     & = \int_{\gamma_1} f(z) \dd{z} + \int_{\gamma_2} f(z) \dd{z}.
  \end{align*}
\end{proof}

\begin{Thm}{Cauchy Integral Formula for Annulus}{}
  Let \(f \in \Hol(A_{R_1, R_2}(z_0))\) and \(0 < R_1 < r_1 < r_2 < R_2\). Then, for all \(z \in A_{r_1, r_2}(z_0)\), we have
  \[
    f(z) = \frac{1}{2\pi i} \int_{C_{r_1}(z_0)} \frac{f(\zeta)}{\zeta - z_0} \dd{\zeta} - \frac{1}{2\pi i} \int_{C_{r_2}(z_0)} \frac{f(\zeta)}{\zeta - z_0} \dd{\zeta}.
  \]
\end{Thm}
\begin{proof}
  For simplicity, assume \(z_0 = 0\). Pick \(\epsilon > 0\) such that \(\overline{B_{\epsilon}(z)} \subseteq A_{r_1, r_2}(0)\). Then \(\zeta \mapsto \frac{f(\zeta)}{\zeta - z}\) is holomorphic on \(A_{r_1, r_2}(0) \setminus \overline{B_{\epsilon}(z)}\). So using Theorem \ref{th:nholedsum} we get,
  \[
    \int_{C_{r_2}(0)} \frac{f(\zeta)}{\zeta - z} \dd{\zeta} = \int_{C_{r_1}(0)} \frac{f(\zeta)}{\zeta - z} \dd{\zeta} + \int_{C_{\epsilon}(z)} \frac{f(\zeta)}{\zeta - z} \dd{\zeta}.
  \]
  By Cauchy Integral Formula, we have
  \[
    f(z) = \frac{1}{2\pi i} \int_{C_{r_2}(0)} \frac{f(\zeta)}{\zeta - z} \dd{\zeta} - \frac{1}{2\pi i} \int_{C_{r_1}(0)} \frac{f(\zeta)}{\zeta - z} \dd{\zeta}.
  \]
  In particular, the same proof gives,
  \[
    f^{(n)}(z) = \frac{n!}{2\pi i} \int_{C_{r_2}(0)} \frac{f(\zeta)}{(\zeta - z)^{n + 1}} \dd{\zeta} - \frac{n!}{2\pi i} \int_{C_{r_1}(0)} \frac{f(\zeta)}{(\zeta - z)^{n + 1}} \dd{\zeta}.
  \]
\end{proof}


\begin{Thm}{}{}
  Let \(f \in \Hol(A_{R_1, R_2}(z_0))\) and \(C = C_r(z_0) \subseteq A\). Then, for all \(z \in C\), we have
  \[
    f(z) = \sum_{n \in \Z} a_n (z - z_0)^n,
  \]
  where
  \[
    a_n = \frac{1}{2 \pi i} \int_{C} \frac{f(\zeta)}{(\zeta - z_0)^{n + 1}} \dd{\zeta}.
  \]
\end{Thm}
\begin{proof}
  Fix \(z \in A\), and let \(z_0 = 0\) for simplicity. We choose \(r_1, r_2 > 0\) such that \(z \in A_{r_1, r_2}(0) \subseteq A_{R_1, R_2}(0)\). Then, by Cauchy Integral Formula for Annulus, we have
  \[
    f(z) = \frac{1}{2\pi i} \int_{C_{r_1}(0)} \frac{f(\zeta)}{\zeta} \dd{\zeta} - \frac{1}{2\pi i} \int_{C_{r_2}(0)} \frac{f(\zeta)}{\zeta} \dd{\zeta}.
  \]
  Note that,
  \[
    \frac{1}{\zeta - z}
    = \begin{cases}
      \frac{1}{\zeta} \sum_{n = 0}^{\infty} \qty(\frac{z}{\zeta})^n, & \abs{z} < \abs{\zeta},  \\
      -\frac{1}{z} \sum_{n = 0}^{\infty} \qty(\frac{\zeta}{z})^n,    & \abs{\zeta} < \abs{z}.
    \end{cases}
  \]
  So, we have
  \[
    f(z) = \sum_{n = 0}^{\infty} \frac{1}{2\pi i} \int_{C_{r_1}(0)} f(\zeta) \frac{1}{\zeta^{n + 1}} \dd{\zeta} - \sum_{n = 0}^{\infty} \frac{1}{2\pi i} \int_{C_{r_2}(0)} f(\zeta) \frac{1}{\zeta^{n + 1}} \dd{\zeta}.
  \]
\end{proof}


\end{document}