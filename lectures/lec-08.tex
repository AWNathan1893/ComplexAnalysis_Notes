\documentclass[../ComplexAnalysis_Notes.tex]{subfiles}

\begin{document}
\chapter*{Lecture 8} %Set chapter name
\addcontentsline{toc}{chapter}{Lecture 8} %Set chapter title
\setcounter{chapter}{8} %Set chapter counter
\setcounter{section}{0}
\setcounter{equation}{0}
\setcounter{figure}{0}

\section{Consequences of analyticity of holomorphic functions}

Recall the main result from the last lecture: if \( f \in \textsf{Hol}(\mathcal{O}) \), where \( \mathcal{O} \) is an open subset of \( \mathbbm{C} \), and \( \overline{B_r(z_0)} \subseteq \mathcal{O} \), then
\[ 
 f(z) = \sum_{n=0}^{\infty} a_n (z-z_0)^n, \quad \forall \, z \in B_r(z_0)
 \]
 where 
 \[ 
  a_n = \frac{f^{(n)}(z_0)}{n!} = \frac{1}{2\pi i} \oint_{C_r(z_0)} \frac{f(\zeta)}{(\zeta-z_0)^{n+1}} \dd{\zeta}.
  \]

We now use focus on some important consequences of this result.

\begin{Thm}{Cauchy's Inequality}{cauchy_ineq}
 Let \( f \in \textsf{Hol}(\mathcal{O}) \) and \( \overline{B_r(z_0)} \subseteq \mathcal{O} \). Then, for all \( n \geq 1 \),
 \[ 
  \abs{f^{(n)}(z_0)} \leq \frac{n!}{r^n} \norm{f}_{C_r(z_0)}\ ,
  \]
  where we define
  \[ 
   \norm{f}_X = \sup_{z \in X} \abs{f}.
   \]
 \end{Thm}

\begin{proof} 
 We have,
 \begin{align*}
  \abs{f^{(n)}(z_0)}
  &= \frac{n!}{2\pi } \abs{\oint_{C_r(z_0)} \frac{f(\zeta)}{(\zeta-z_0)^{n+1}} \dd{\zeta}} \\
  &\leq \frac{n!}{2\pi } \sup_{z \in C_r(z_0)} \abs{\frac{f(z)}{(z-z_0)^{n+1}}} \ell(C_r(z_0)) \tag{triangle inequality}
 \end{align*}
 We now use the fact that \( \abs{z-z_0} = r \) for \( z \in C_r(z_0) \) and \( \ell(C_r(z_0)) = 2\pi r \) to get
 \[ 
  \abs{f^{(n)}(z_0)} \leq \frac{n!}{r^n} \norm{f}_{C_r(z_0)}
  \]
  as was required.
 \end{proof}

\begin{Thm}{Liouville's theorem}{liouv}
 Let \( f \) be a bounded and entire function, i.e, \( f \in \textsf{Hol}(\mathbbm{C}) \). Then, \( f \) is constant.
 \end{Thm}

\begin{proof} 
 We use Theorem \ref{th:cauchy_ineq} for \( f' \). Fix \( z_0 \in \mathbbm{C} \) and \( r > 0 \). As \( \overline{B_r(z_0)} \subseteq \mathbbm{C} \), we have
 \[ 
  \abs{f'(z_0)} \leq \frac{1}{r} \norm{f}_{C_r(z_0)} \leq \frac{\norm{f}_{\infty}}{r}.
  \]
  As \( r \) is an arbitrary positive real, we get \( f'(z_0) = 0 \) for all \( z_0 \in \mathbbm{C} \). Hence, \( f \) is constant.
 \end{proof}
\pagebreak

\textbf{Remarks}: \begin{enumerate}[label = (\arabic*)]
  \item As \( \sin z, \cos z \) are non-constant entire functions, they are not bounded on \( \mathbbm{C} \).
  \item The theorem is false for \( \mathbbm{R}^n \), as there are analytic functions (e.g, \( \sin x \)) that are bounded and non-constant.
\end{enumerate}

We now give a rather slick proof of one of the most useful and ubiquitous results in all of ,mathematics, the fact that \( \mathbbm{C} \) is algebraically closed.

\begin{Thm}{Fundamental Theorem of Algebra}{}
 Let \( p \in \mathbbm{C}[z] \) be non-constant. Then, there exists \( z_0 \in \mathbbm{C} \) such that \( p(z_0) = 0 \).
 \end{Thm}

\begin{proof} 
 Suppose no such \( z_0 \) exists. Then, \( \frac{1}{p} \) is an entire function. If \( p(z) = \sum_{n=0}^{d}a_nz^n \) for complex numbers \( a_n \), then we get
 \[ 
  \frac{1}{p(z)} = \frac{1}{z^d} \frac{1}{\sum_{n=0}^{d}a_{n}z^{n-d}} \xrightarrow{z \to \infty} 0
  \]
  Therefore, \( \frac{1}{p} \) is bounded for \( \abs{z} > R \), for some \( R > 0 \), and as \( \frac{1}{p} \) is entire and in particular continuous, it attains a finite supremum on \( \overline{B_R(0)} \). So, \( \frac{1}{p} \) is a bounded entire function, and hence must be constant by Theorem \ref{th:liouv}. This contradicts the assumption that \( p \) is non-constant, and hence, some root \( z_0 \) of \( p \) exists in \( \mathbbm{C} \).
 \end{proof}

Recall Theorem \ref{thm:4.2} that if \( f \in C(\mathcal{U}) \) for some open convex set \( \mathcal{U} \) and \( \oint_{\partial\Delta} f = 0 \) for all \( \Delta \subseteq \mathcal{U} \), then there exists \( g \in \textsf{Hol}(\mathcal{U}) \) such that \( g' = f \). One corollary of this is that any such function \( f \) must be holomorphic on the set \( \mathcal{U} \). A stronger version of this result is the following.

\begin{Thm}{Morera's theorem}{morera}
 Let \( \mathcal{O} \subseteq \mathbbm{C} \) be any open set, and \( f \in C(\mathcal{O}) \). Then, \( f \in \textsf{Hol}(\mathcal{U}) \) if and only if \( \oint_{\partial \Delta} f = 0 \) for all \( \Delta \subseteq \mathcal{O} \).
 \end{Thm}

\begin{proof} 
 If \( f \in \textsf{Hol}(\mathcal{O}) \), we get \( \oint_{\partial \Delta} f = 0 \) for all \( \Delta \subseteq \mathcal{O} \) by Cauchy's integral theorem. 
 \smallskip

 Conversely, suppose that \( \oint_{\partial \Delta} f = 0 \) for all \( \Delta \subseteq \mathcal{O} \). Then, as any \( \Delta \subseteq \mathcal{O} \) is a closed convex set, we get an open convex set \( \mathcal{U} \) containing \( \Delta \) which is also contained in \( \mathcal{O} \). Over \( \mathcal{U} \), we apply Theorem \ref{thm:4.2} and get a local primitive of \( f \). In particular, \( f \in \textsf{Hol}(\mathcal{U}) \). Therefore, as \( f \) is holomorphic at each point of \( \mathcal{O} \), we get \( f \in \textsf{Hol}(\mathcal{O}) \) and so we are done.
 \end{proof}

\begin{Thm}{}{hol_unif}
 \, Let \( \qty{f_n}_{\mathbbm{N}} \subseteq \textsf{Hol}(\mathcal{O}) \). Suppose \( f_n \to f \)  uniformly. Then, \( f \in \textsf{Hol}(\mathcal{O}) \).
 \end{Thm}

\textbf{Remarks}:\begin{enumerate}[label = (\arabic*)]
  \item We define \( f_n \to f \) uniformly on an open set, if the convergence is uniform on any compact subset of the open set.
  \item The result fails miserably for \( \mathbbm{R} \)! For example, the Weierstrass approximation theorem says \( \overline{\mathbbm{R}[x]} \simeq C([0,1], \mathbbm{R}) \), that is, \textit{any} continuous function can be approximated uniformly by polynomial functions, which are not just smooth but in fact their derivatives vanish after some finite order.
\end{enumerate}

\begin{proof} 
By uniform convergence, \( f \in C(\mathcal{O}) \). Fix \( \Delta \subseteq \mathcal{O} \). As \( f_n \in \textsf{Hol}(\mathcal{O}) \), we get by Morera's theorem that \( \oint_{\partial\Delta} f = 0 \). But,
\[ 
 \oint_{\partial\Delta} f = \oint_{\partial\Delta} \lim_{n \to \infty} f_n = \lim_{n \to \infty} \oint_{\partial\Delta} f_n = 0
 \]
 and so by Morera's theorem again, \( f \in \textsf{Hol}(\mathcal{O}) \).
\end{proof}

\begin{Thm}{}{}
 \, Let the same assumptions hold as Theorem \ref{th:hol_unif}. Then, for all \( k \geq 1 \),
 \[ 
  f_n^{(k)} \to f^{(k)} 
  \]
  uniformly on \( \mathcal{O} \).
 \end{Thm}

\begin{proof} 
 It is clearly enough to show the result for \( k=1 \), that is, \( f_n' \to f' \) uniformly. Further, it is enough to show that \( f_n' \to f' \) uniformly on the closed discs \( \overline{B_r(z_0)} \subseteq \mathcal{O} \). Fix \( r, z_0 \) and \( \delta \) small such that
 \[ 
  \overline{B_r(z_0)} \subseteq \overline{B_{r+\delta}(z_0)} \subseteq \mathcal{O}.
  \]
  Consider \( z \in \overline{B_r(z_0)} \) and let \( C \) be a circle of radius \( R \) centered at \( z \). Assume \( \frac{\delta}{2} < R < \delta \) so that \( C \subseteq \mathcal{O} \). By Theorem \ref{th:cauchy_ineq},
  \begin{align*}
     \abs{(f_n - f)'(z)} 
     &\leq \frac{1}{R}\norm{f_n-f}_C \\
     \implies \abs{(f_n - f)'(z)} 
     &< \frac{1}{\delta} \norm{f_n-f}_C \\
     \implies \abs{(f_n - f)'(z)} 
     &< \frac{1}{\delta} \norm{f_n-f}_{\overline{B_{r+\delta}(z_0)}} \\
     \implies \norm{f_n'-f'}_{\overline{B_r(z_0)}}
     &\leq \frac{1}{\delta} \norm{f_n-f}_{\overline{B_{r+\delta}(z_0)}}
  \end{align*}
  As \( f_n \to f \) uniformly, we get that \( f_n' \to f' \) uniformly from the above inequality.
 \end{proof}

\section{Zeroes of Analytic Functions}
Let \( \mathcal{O} \) be an open subset of \( \mathbbm{C} \), \( f \in \textsf{Hol}(\mathcal{O}) \). We define the set of zeroes of \( f \),
\[ 
 \mathcal{Z}(f) = \qty{z \in \mathcal{O} \mid f(z) = 0}
 \]
Let \( z_0 \in \mathcal{Z}(f) \). Consider the power series of \( f \) near \( z_0 \),
\[ 
 f(z) = \sum_{n\geq m} a_n (z-z_0)^n
 \]
We have the following two cases:
\begin{enumerate}[label = (\roman*)]
  \item There exists \( m_0 \in \mathbbm{N} \) such that \( a_n = 0 \) for all \( n < m_0 \) and \( a_{m_0} \neq 0 \).
  \smallskip

  \begin{Def}{Order of a zero}{}
   In this case, we define that \( z \) is a zero of \( f \) of (finite) order \( m_0 \), and write \( \textsf{Ord}(f;z_0) = m_0 \).
   \end{Def}

   We have in this case, for \( z \in B_r(z_0) \),
   \[ 
    f(z) = (z-z_0)^{m_0}g(z),
    \]
where \( g(z_0) \neq 0 \) and \( g \in \textsf{Hol}(B_r(z_0)) \). As \( g \in C(B_r(z_0)) \) in particular, there is \( r' \leq r \) such that \( g(z) \neq 0 \) for all \( z \in B_{r'}(z_0) \). Therefore, \( f(z) \neq 0 \) for all \( z \in B_{r'(z_0)} \setminus \qty{z_0} \). We have thus proved the following theorem:

\begin{Thm}{}{}
 \, Let \( f \in \textsf{Hol}(\mathcal{O}) \) for \( \mathcal{O} \) a domain. If \( z_0 \in \mathcal{Z}(f) \) is a finite order zero of \( f \), \( z_0 \) is an isolated point of \( \mathcal{Z}(f) \).
 \end{Thm}
\pagebreak
 
 \item \( a_n = 0 \) for all \( n \geq 0 \). 
 \smallskip

 \begin{Def}{}{}
  In this case, we define that \( z \) is a zero of \( f \) of infinite order.
  \end{Def}

  We have in this case, \( f \equiv 0 \) on \( B_r(z_0) \) for some \( r > 0 \). This proves the following theorem.
  \begin{Thm}{}{}
   \, The set
   \[ 
    \widetilde{\mathcal{O}} = \qty{z \in \mathcal{Z}(f) \mid z \text{ an infinite zero of } f}
    \]
    is open.
   \end{Thm}
\end{enumerate}

The following theorem is an easy consequence of the above results but it is a immensely useful result by itself.

\begin{Thm}{Nature of zeroes}{}
 \, Let \( f \in \textsf{Hol}(\mathcal{O}) \) with \( \mathcal{O} \) a domain in \( \mathbbm{C} \) and suppose \( f \) is not identically zero. Then all zeroes of \( f \) are of finite order and are isolated.
 \end{Thm}

 \begin{proof} 
  We know \( \widetilde{\mathcal{O}} \) is open. We will show that it is also closed. Consider \( z_0 \in \mathcal{O} \setminus \widetilde{\mathcal{O}} \). Then, there is \( m \geq 0 \) such that \( f^{(m)}(z_0) \neq 0 \). By continuity, there is \( R > 0 \) such that \( f^{(m)}(z) \neq 0 \) for all \( z \in B_R(z_0) \subseteq \mathcal{O} \setminus \widetilde{\mathcal{O}} \). Hence, \( \widetilde{\mathcal{O}} \) is a clopen set, and as it is not the full set \( \mathcal{O} \), it must be empty. Therefore, all zeroes of \( f \) are of finite order, and they are isolated by the above results.
  \end{proof}

\begin{Cor}{}{}
 Let \( f \in \textsf{Hol}(\mathcal{O}) \), \( \mathcal{O} \) a domain. If \( \mathcal{Z}(f) \) has a limit point in \( \mathcal{O} \), then \( f \equiv 0 \).
 \end{Cor}

 The above result can be restated as follows:
 \begin{Thm}{Identity theorem}{ident}
  \, Suppose \( f,g \in \textsf{Hol}(\mathcal{O}) \), \( \mathcal{O} \) a domain. If \( f = g \) on \( X \subseteq \mathcal{O} \) which has a limit point in \( \mathcal{O} \), then \( f = g \) on all of \( \mathcal{O} \).
  \end{Thm}

\end{document}